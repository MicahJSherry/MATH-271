\documentclass{article}


\usepackage[english]{babel}
\usepackage{listings}
\usepackage[letterpaper,top=2.5cm,bottom=2.5cm,left=2.5cm,right=2.5cm,marginparwidth=1.75cm]{geometry}

\usepackage{amsmath,amsthm,amssymb}
\usepackage{bbold}
\usepackage{graphicx}
\usepackage{listings}
\usepackage{multicol}
\usepackage{setspace}
\usepackage{enumitem}


%common set symbols 
\newcommand{\Z}{\mathbb{Z}}
\newcommand{\N}{\mathbb{N}}
\newcommand{\R}{\mathbb{R}}
\newcommand{\Q}{\mathbb{Q}}
\newcommand{\C}{\mathbb{C}}
\newcommand{\nullset}{\varnothing}
\newcommand{\powerset}[1]{\mathcal{P}({#1})}


\newcommand{\st}{\text{ such that }}
\newcommand{\pmi}{\text{principle of mathematical induction}}
\newcommand{\nitem}[1] % sets enumerate to a specific number {arg1}
{
	\setcounter{enumi}{#1}
	\addtocounter{enumi}{-1}
	\item
}

\title{MATH 271: Chapter 9 homework}
\author{Micah Sherry}
\singlespacing

\begin{document}
	\maketitle
	
	\begin{enumerate}
	\nitem{4} If $n \in \N$ Then $ 1*2+2*3+3*4+4*5+\hdots+n(n+1)= \frac{n(n+1)(n+2)}{3}$ 
	\begin{proof} We will prove this with Induction. \\
		\textbf{Base Case:}
		Consider the base case $n=1$, $$\sum_{i=1}^{1}(i)(i+1)= 1*2=2$$ 
		and $$\frac{n(n+1)(n+2)}{3}= \frac{1(2)(3)}{3}=2$$ 
		Therefore the base case holds.
		
		\textbf{Induction Hypothesis:}
		Now Assume that $1*2+2*3+3*4+4*5+\hdots+k(k+1)= \frac{k(k+1)(k+2)}{3}$ is true for some $k \in \N$.
		
		\textbf{Inductive step:} Now consider the case when $n=k+1$.
		\begin{align*}
			\sum_{i=1}^{k+1}(i)(i+1) &= \sum_{i=1}^{k}(i)(i+1) +(k+1)(k+2) \\
			 	&=\frac{k(k+1)(k+2)}{3}+(k+1)(k+2) & \text{(By the Induction Hypothesis)}\\
			 	&=(\frac{k}{3}+1)(k+1)(k+2) & \text{(By factoring)}	 	\\
				&=\frac{(k+1)(k+2)(k+3)}{3} & \text{(By factoring and rearranging)}
		\end{align*}
		Since the base case holds and $S_k \implies S_{k+1}$ by the \pmi the statement is true $\forall n \in \N$
		
	\end{proof}
	
	\nitem{8} If $n \in \N$ then $\frac{1}{2!}+ \frac{2}{3!}+ \frac{3}{4!}+\hdots \frac{n}{(n+1)!}=1- \frac{1}{(n+1)!}$ 
	\begin{proof} We will prove this with Induction. \\
		\textbf{Base Case:}
		
		\textbf{Induction Hypothesis:}
	
		\textbf{Inductive step:} Now consider the case when $n=k+1$.
		
	\end{proof}
	\end{enumerate}
	
\end{document} 