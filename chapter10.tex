\documentclass{article}


\usepackage[english]{babel}
\usepackage{listings}
\usepackage[letterpaper,top=2.5cm,bottom=2.5cm,left=2.5cm,right=2.5cm,marginparwidth=1.75cm]{geometry}

\usepackage{amsmath,amsthm,amssymb}
\usepackage{bbold}
\usepackage{graphicx}
\usepackage{listings}
\usepackage{multicol}
\usepackage{setspace}
\usepackage{enumitem}


%common set symbols 
\newcommand{\Z}{\mathbb{Z}}
\newcommand{\N}{\mathbb{N}}
\newcommand{\R}{\mathbb{R}}
\newcommand{\Q}{\mathbb{Q}}
\newcommand{\C}{\mathbb{C}}
\newcommand{\nullset}{\varnothing}
\newcommand{\powerset}[1]{\mathcal{P}({#1})}
\newcommand{\ol}[1]{\overline{#1}}

\newcommand{\st}{\text{ such that }}
\newcommand{\pmi}{\text{ principle of mathematical induction }}
\newcommand{\nitem}[1] % sets enumerate to a specific number {arg1}
{
	\setcounter{enumi}{#1}
	\addtocounter{enumi}{-1}
	\item
}

\title{MATH 271: Chapter 10 homework}
\author{Micah Sherry}
\singlespacing

\begin{document}
	\maketitle
	
	\begin{enumerate}
		\nitem{4} If $n \in \N$ Then $ 1*2+2*3+3*4+4*5+\hdots+n(n+1)= \frac{n(n+1)(n+2)}{3}$ 
		\begin{proof} We will prove this with Induction. \\
			\textbf{Base Case:}
			Consider the base case $n=1$, $$\sum_{i=1}^{1}(i)(i+1)= 1*2=2$$ 
			and $$\frac{n(n+1)(n+2)}{3}= \frac{1(2)(3)}{3}=2$$ 
			Therefore the base case holds.
			
			\textbf{Induction Hypothesis:}
			Now Assume that $1*2+2*3+3*4+4*5+\hdots+k(k+1)= \frac{k(k+1)(k+2)}{3}$ is true for some $k \in \N$.
			
			\textbf{Inductive step:} Now consider the case when $n=k+1$.
			\begin{align*}
				\sum_{i=1}^{k+1}(i)(i+1) &= \sum_{i=1}^{k}(i)(i+1) +(k+1)(k+2) \\
				&=\frac{k(k+1)(k+2)}{3}+(k+1)(k+2) & \text{(By the Induction Hypothesis)}\\
				&=(\frac{k}{3}+1)(k+1)(k+2) & \text{(By factoring)}	 	\\
				&=\frac{(k+1)(k+2)(k+3)}{3} & \text{(By factoring and rearranging)}
			\end{align*}
			Since the base case holds and $S_k \implies S_{k+1}$ by the \pmi, the statement is true $\forall n \in \N$
			
		\end{proof}
		
		\nitem{8} If $n \in \N$ then $\frac{1}{2!}+ \frac{2}{3!}+ \frac{3}{4!}+\hdots +\frac{n}{(n+1)!}=1- \frac{1}{(n+1)!}$ 
		\begin{proof} We will prove this with Induction. \\
			\textbf{Base Case:} 
			Consider the base case $n=1$,
			$$\sum_{i=1}^{1}\frac{i}{(i+1)!}= \frac{1}{2!}=\frac{1}{2}$$   
			and
			$$1-\frac{1}{(n+1)!} = 1 - \frac{1}{2!}=\frac{1}{2}$$
			So the base case holds.
			
			\textbf{Induction Hypothesis:} 
			Now assume that $\frac{1}{2!}+ \frac{2}{3!}+ \frac{3}{4!}+\hdots+ \frac{k}{(k+1)!}=1- \frac{1}{(k+1)!}$ is true for some $k \in \N$.
			
			\textbf{Inductive step:} Now consider the case when $n=k+1$.
			\begin{align*}
				\sum_{i=1}^{k+1}\frac{i}{(i+1)!} &= \sum_{i=1}^{k}\frac{i}{(i+1)!}+ \frac{k+1}{(k+2)!} \\
				&=1- \frac{1}{(k+1)!} + \frac{k+1}{(k+2)!} &\text{(by the induction hypothesis)}\\
				&=1+ \frac{-k-2 + k+1}{(k+2)!} &\text{(by factoring)}\\
				&=1- \frac{1}{(k+2)!}
			\end{align*}
			
			Since the base case holds and $S_k \implies S_{k+1}$ by the \pmi,the statement is true $\forall n \in \N$
		\end{proof}
		
		\nitem{10} prove that $3|5^{2n}-1$ for ever integer $n \ge 0$ 
		\begin{proof}
			\textbf{Base Case:} 
			Consider the base case $n=0$.
			Then $5^{2n}-1 =0$. Since 0 is divisible by all integers the base case holds.
			
			\textbf{Induction Hypothesis:} Assume there is a $k$ such that $3|5^{2k}-1$. (Note $5^{2k} = 3m +1$ for some $m \in \Z$)\\
			\textbf{Inductive step:} Now consider the case when $n=k+1$. So,
			\begin{align*}
				5^{2k+2}-1 &= 5^{2k}5^2 -1\\
				&= (3m+1)5^2 -1 &\text{(by the Induction Hypothesis)}\\
				&= 3(25m)+24 
				&= 3(25m+8)
			\end{align*}
			which implies $3|5^{2k+2}-1$. Since the base case holds and $S_k \implies S_{k+1}$ 
			by the principle of mathematical induction, the statement is true $\forall n, n \ge 0$
		\end{proof}
		
		\nitem{18} Suppose $A_1, A_2, \hdots A_n$ are sets in a universal set $U$ and $n \ge 2$. 
		Prove that $$\ol{A_1 \cup A_2 \cup \hdots \cup A_n} =\ol{A_1} \cap \ol{A_2} \cap \hdots \cap \ol{A_n}$$
		\begin{proof}
			\textbf{Base Case:} 
			Consider the base case $n=2$.
			\begin{align*}
				\ol{A_1 \cup A_2} &=\{x \in U|\sim(x \in A_1 \cup A_2)\} & \text{(by definition of complement)}\\
				&=\{x \in U|\sim(x \in A_1 \vee x \in A_2)\} & \text{(by definition of union)}\\
				&=\{x \in U|\sim(x \in A_1) \wedge \sim(x \in A_2)\} & \text{(by DeMorgans law)}\\
				&=\{x \in U|x \in\ol{A_1} \wedge x \in\ol{A_2} \} & \text{(by definition of complement)}\\
				&=\{x \in U|x \in\ol{A_1} \cap \ol{A_2} \} & \text{(by definition of intersection)}
				&= \ol{A_1} \cap \ol{A_2}
			\end{align*}
			Therefore the base case holds.
			
			\textbf{Induction Hypothesis:} $\ol{A_1 \cup A_2 \cup \hdots \cup A_k} =\ol{A_1} \cap \ol{A_2} \cap \hdots \cap \ol{A_k}$ is true for some $k \ge 2$
			
			\textbf{Inductive step:} Now consider the case when $n=k+1$. 
			Let B denote $A_1 \cup A_2 \cup \hdots \cup A_{k}$. by the inductive hypothesis $\ol{B} =\ol{A_1} \cap \ol{A_2} \cap \hdots \cap \ol{A_k}$. 
			\begin{align*}
				\ol{B \cup A_{k+1}} &=\{x \in U|\sim(x \in B \cup A_{k+1})\} & \text{(by definition of complement)}\\
				&=\{x \in U|\sim(x \in B \vee x \in A_{k+1})\} & \text{(by definition of union)}\\
				&=\{x \in U|\sim(x \in B) \wedge \sim(x \in A_{k+1})\} & \text{(by DeMorgans law)}\\
				&=\{x \in U|x \in\ol{B} \wedge x \in \ol{A_{k+1}} \} & \text{(by definition of complement)}\\
				&=\{x \in U|x \in\ol{B} \cap \ol{A_{k+1}} \} & \text{(by definition of intersection)}\\
				&=\ol{B} \cap \ol{A_{k+1}}\\
				&= \ol{A_1} \cap \ol{A_2} \cap \hdots \cap \ol{A_{k+1}}\\
			\end{align*} 
			Since the base case holds and $S_k \implies S_{k+1}$ 
			by the principle of mathematical induction, the statement is true $\forall n, n \ge 2$
		\end{proof}
		
		\nitem{26}  Concerning the Fibonacci sequence, prove that 
		$$\sum_{i=1}^{n}F_i^2= F_n F_{n+1}$$
		\begin{proof}
			\textbf{Base Case:} 
			Consider the base case $n=1$.
			$$\sum_{i=1}^{1}F_i^2=F_1^2 = 1 \text{ and } F_{1}F_{2} = 1*1 =1 $$
			therefore the base case holds
			
			\textbf{Induction Hypothesis:} Assume $$\sum_{i=1}^{k}F_i^2= F_k F_{k+1}$$ is True for some k.
			
			\textbf{Inductive step:} Now consider the case when $n=k+1$.
			\begin{align*}
				\sum_{i=1}^{k+1}F_i^2 &= \sum_{i=1}^{k}F_i^2 + F_{k+1}^2\\
				&= F_k F_{k+1} + F_{k+1}^2 & \text{(by the induction hypothesis)}\\
				&= F_{k+1}(F_k  + F_{k+1}) & \text{(by factoring)}\\
				&= F_{k+1}(F_{k+2}) & \text{(by definition of the Fibonacci sequence)}\\
			\end{align*}
			Since the base case holds and $S_k \implies S_{k+1}$ 
			by the principle of mathematical induction, the statement is true $\forall n \in \N$
		\end{proof}
		
		\nitem{34} prove that $3^1 + 3^2+ 3^3 + 3^4\hdots+3^n = \frac{3^{n+1}-3}{2}$ for every $n \in \N$.
		\begin{proof}
			\textbf{Base Case:} Consider the base case $n=1$.
			$3^1 = 3 \text{ and } \frac{3^2-3}{2}$
			So the base Case holds.
			
			\textbf{Induction Hypothesis:} Assume $3^1 + 3^2+ 3^3 + 3^4\hdots+3^k = \frac{3^{k+1}-3}{2}$ for some $k \in \N$
			
			\textbf{Inductive step:} Now consider the case when $n=k+1$.
			\begin{align*}
				\sum_{i=1}^{k+1}3^i &= \sum_{i=1}^{k}3^i +3^{k+1} \\
				&= \frac{3^{k+1}-3}{2} +3^{k+1} &\text{(by the induction hypothesis)}\\
				&= \frac{3^{k+1}-3+2(3^{k+1})}{2} \\
				&= \frac{3^{k+1}(1+2)-3}{2}\\
				&= \frac{3^{k+2}-3}{2}
			\end{align*}
			Since the base case holds and $S_k \implies S_{k+1}$ 
			by the principle of mathematical induction, the statement is true $\forall n \in \N$
		\end{proof}
		
	\end{enumerate}
	
\end{document} 