\documentclass{article}


\usepackage[english]{babel}
\usepackage{listings}
\usepackage[letterpaper,top=2.5cm,bottom=2.5cm,left=2.5cm,right=2.5cm,marginparwidth=1.75cm]{geometry}

\usepackage{amsmath,amsthm,amssymb}
\usepackage{bbold}
\usepackage{graphicx}
\usepackage{listings}
\usepackage{multicol}
\title{MATH 271: chapter 7 homework}
\author{Micah Sherry}

\newcommand{\ints}{\mathbb{Z}}
\newcommand{\reals}{\mathbb{R}}
\newcommand{\st}{\text{ such that }}
\newcommand{\nullset}{\varnothing}


\begin{document}
	\maketitle
	
	\begin{enumerate}
		\setcounter{enumi}{3}
		\item Given an integer $a$, then $a^2 +4a+5$ is odd if and only if $a$ is even.
		
		\begin{proof}
			First, we will show that: if $a$ is even, then $a^2 +4a+5$ is odd and we will show this directly.
			Assuming $a$ is even, then $\exists n \in \ints \st a=2n$. 
			So, 
			\begin{align*}
				a^2 + 4a + 5 &= (2n)^2 + 4(2n) + 5\\
				&= 4n^2 + 8n + 5\\
				&= 2(2n^2 + 4n + 2) + 1
			\end{align*}
			Since  $2(2n^2 + 4n + 2) + 1$ is odd (by definition of odd) the implication holds.
			
			Next, we will show that: 
			If $a^2 +4a+5$ is odd, then a is even.
			Consider the contra-positive of this statement: 
			If a is odd, then $a^2 +4a+5$ is even.
			Assuming $a$ is odd, then, $\exists n \in \ints \st a=2n+1$. 
			So, 
			\begin{align*}
				a^2 + 4a + 5 &= (2n+1)^2 + 4(2n+1) + 5\\
				&=  4n^2+4n+1 +8n + 4 + 5\\
				&=  4n^2+12n +10\\
				&=  2(2n^2+6n +5)
			\end{align*}
			Since $2(2n^2+6n +5)$ is even the contra-positive holds and so does the original implication.
			
			Since $a$ is even, implies $a^2 +4a+5$ is odd and $a^2 +4a+5$ is odd, implies $a$ is even. The original statement holds.
			
			
		\end{proof}
		
		\setcounter{enumi}{5}
		\item Suppose $x, y \in \reals $. Then  $x^3 + x^2y=y^2 +xy$ if and only if $y = x^2 $ or $y = -x$ 
		\begin{proof}
			First we will show that: if $y = x^2$ or $y = -x$, then $x^3 + x^2y=y^2 +xy$.
			
			\textbf{case 1:} $y = -x$ \\ 
			\begin{align*}
				x^3+x^2y &= x^3 + x^2(-x) & \text{(by substitution)}\\
				&=  x^3 - x^3 \\
				&= 0\\
			\end{align*}
			and
			\begin{align*}
				y^2 +xy&= (-x)^2+x(-x) & \text{(by substitution)}\\
				&=  x^2 - x^2 \\
				&= 0\\
			\end{align*}
			Therefore $x^3+x^2y = y^2 +xy$ when $y = -x$.
			
			\textbf{case 2:} $y = x^2$ \\
			Consider $x^4 + x^3$, 
			\begin{align*}
				x^4+x^3 &= x^4 + x^3\\
				&= (x^2)^2+(x^2)x & \text{(by factoring)}\\
				&= y^2+yx & \text{(by substitution)}\\
				\\
				x^4+x^3 &= x^4 + x^3\\
				&= x^2(x^2)+x^3 & \text{(by factoring)}\\
				&= x^2y+x^3 & \text{(by substitution)}.
			\end{align*} 
			Thus, $x^3+x^2y =y^3+xy$ when $y=x^2$. 
			since both cases hold so does the implication.
			
			next consider the implication: if $x^3 + x^2y=y^2 +xy$, then  $y = x^2$ or $y = -x$.
			Assume for the sake of contradiction that:
			$x^3 + x^2y=y^2 +xy$ and $y \neq x^2$ and $y \neq -x$. 
			Since,
			\begin{align*}
				x^3 + x^2y &= y^2 +xy\\
				x^2(x+y) &= y(x+y) & \text{(by factoring)}\\
				\frac{x^2(x+y)}{(x+y)} &= \frac{y(x+y)}{(x+y)} & \text{(since } y \neq -x\text{)}\ \\
				x^2 &= y.
			\end{align*}
			Which is a contradiction therefore the implication holds.
			
			Since, $y = x^2$ or $y = -x$, implies $x^3 + x^2y=y^2 +xy$ and $x^3 + x^2y=y^2 +xy$, implies $y = x^2$ or $y = -x$ the original statement holds
			
		\end{proof}
		
		
		
		\setcounter{enumi}{7}
		\item Suppose $a,b \in \ints$. Prove that $a \equiv b \pmod{10}$ if and only if  $a \equiv b \pmod{2}$ and $a \equiv b \pmod{5}$.
		\begin{proof}
			First we will show that: if  $a \equiv b \pmod{2}$ and $a \equiv b \pmod{5}$, then  $a \equiv b \pmod{10}$. Assuming $a \equiv b \pmod{2}$ and $a \equiv b \pmod{5}$. So, $2|(a-b)$ (by definition of mod) and $5|(a-b)$ (by definition of mod). Therefore  $10|(a-b)$ because 2 and 5 share no factors. Thus $a \equiv b \pmod{10}$ So, the implication holds  
			
			Next we will show that: if $a \equiv b \pmod{10}$, then  $a \equiv b \pmod{2}$ and $a \equiv b \pmod{5}$.\\
			Assuming $a \equiv b \pmod{10}$, then $10|(a-b)$. So, $\exists n \in \ints \st 10n=(a-b)$. 
			So, 
			\begin{align*}
				(a-b) &= 10n &\\
				&= 2(5n)
			\end{align*}
			and 
			\begin{align*}
				(a-b) &= 10n&\\
				&= 5(2n)
			\end{align*}
			So $2|(a-b)$ and $5|(a-b)$. Therefore $a \equiv b \pmod{2}$ and $a \equiv b \pmod{5}$ thus the implication holds. 
			since $a \equiv b \pmod{2}$ and $a \equiv b \pmod{5}$, implies  $a \equiv b \pmod{10}$. and 
			$a \equiv b \pmod{10}$, implies  $a \equiv b \pmod{2}$ and $a \equiv b \pmod{5}$,the  original statement holds
		\end{proof}
	\end{enumerate}
	
\end{document}

