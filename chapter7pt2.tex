\documentclass{article}


\usepackage[english]{babel}
\usepackage{listings}
\usepackage[letterpaper,top=2.5cm,bottom=2.5cm,left=2.5cm,right=2.5cm,marginparwidth=1.75cm]{geometry}

\usepackage{amsmath,amsthm,amssymb}
\usepackage{bbold}
\usepackage{graphicx}
\usepackage{listings}
\usepackage{multicol}
\usepackage{setspace}
\usepackage{enumitem}


\newcommand{\Z}{\mathbb{Z}}
\newcommand{\N}{\mathbb{N}}
\newcommand{\R}{\mathbb{R}}
\newcommand{\st}{\text{ such that }}
\newcommand{\nullset}{\varnothing}
\newcommand{\powerset}[1]{\mathcal{P}({#1})}



\title{MATH 271: Chapter 7 homework}
\author{Micah Sherry}
\singlespacing

\begin{document}
	\maketitle
	
	\begin{enumerate}
		\setcounter{enumi}{11}
		\item There exists a positive real number $x$ for which $x^2 <\sqrt{x}$.
			\begin{proof} 
				Consider the case when $x = \frac{1}{4}$. Then 
				$x^2= (\frac{1}{4})^2 = \frac{1}{16}$  and $\sqrt{x} = \sqrt{\frac{1}{4}} =\frac{1}{2}$. Since $\frac{1}{16} < \frac{1}{2}$. The proposition holds.
			\end{proof}
		\setcounter{enumi}{15}
		\item Suppose $a,b \in \Z$. If $ab$ is odd, then $a^2+b^2$ is even.
		\begin{proof}
			assume that $ab$ is odd. Then $a$ and $b$ are both odd. So, $\exists n,m \in \Z \st a= 2m+1$ and $ b=2n+1$.
			\begin{align*}
				a^2+b^2 &= (2m+1)^2 +(2n+1)^2 		& \text{(by substitution)}\\
						&= 4m^2+4m + 1 +4n^2+4n + 1 \\
						&= 2(2m^2+2n^2+2m+2n+1).
			\end{align*}
			Since, $2(2m^2+2n^2+2m+2n+1)$ is even by definition, The proposition holds.
		\end{proof}
		
		\setcounter{enumi}{17}
		\item There is a set $X$ for which $\Z \in X$ and $\Z \subseteq X$.
		\begin{proof}
			Consider the set $X = \Z \cup \{\Z\}$. Therefore, $X = \{x| x \in \Z \text{ or } x \in \{\Z\}  \}$ (by definition of union). So $\Z \in X$ because $\Z \in \{Z\}$. And $\Z$ is a subset of $X$ because $\Z$ is a subset of itself.  Since $\Z$ is a subset and an element of $X$, the proposition holds
		\end{proof} 
		
		\setcounter{enumi}{19}
		\item There exists an $n \in N$ for which $11 | (2n - 1)$.
		\begin{proof}
			Consider the case where $n=10$. Then $2^n-1 = 1023 = 11(93)$. So, 11 divides $2^{10}-1$. Therefore the proposition holds. (Note: I found this solution with a Python script but I think an argument can be made that $(n+1)|(2^n-1)$).
		\end{proof}
		
		\setcounter{enumi}{21}
		\item  if $n \in \Z$ then $4|n^2$ or $4|(n^2-1)$
		\begin{proof}
			Notice there are 2 cases for n: odd and even). 
			\begin{enumerate}[label={ Case \arabic*:}]
				\item n is odd. So, $\exists k \in \Z  \st n= 2k+1$. Consider,
					\begin{align*}
						n^2-1 &= (2k+1)^2-1\\
							  &= 4k^2+ 4k +1-1\\
							  &= 4(k^2+k).
					\end{align*}
					So $4$ divides $n^2-1$ when $n$ is odd. 
				\item n is even. So, $\exists k \in \Z  \st n= 2k$. Consider,
				\begin{align*}
					n^2 &= (2k)^2\\
					&= 4(k^2).
				\end{align*}
				So $4$ divides $n^2$ when $n$ is even. Since 4 divides $n^2$ or $n^2-1$ the proposition holds.
			\end{enumerate}
		\end{proof}
		
		
		\setcounter{enumi}{29}
		\item  Suppose $a,b,p \in \Z$ and $p$ is prime. Prove that if $p|ab$ then $p|a$ or $p|b$.
		\begin{proof}
			Assume for the sake of contradiction  that $p|ab$ and $p\nmid a$ and $p \nmid b$. So, $\exists k \st ab = pk.$ And the $\gcd(p, b)=1,$ because $p\nmid b$ and p only has factors 1 and p. so by proposition 7.1 (page 152) $\exists m, n \st 1 = bm + pn$.
			\begin{align*}
				1 &= bm + pn\\
				a &= abm + pna & \text{(multiplying both side by $a$)}\\
				  &= pkm + pna & \text{(by substituting ab for pk)}\\
				  &= p(km+na) 				 
			\end{align*}
			Therefore $p|a$ which is a contradiction therefore the original statement holds.
		\end{proof}
		
		
		\item if $\gcd(a,c)= \gcd(b,c)=1$, then $\gcd(ab,c)=1$.
		\begin{proof}
			Assume $\gcd(a,c)= \gcd(b,c)=1$, So, $\exists w, x, y, z  \in \Z \st 1= aw+ cx$ and $1= by + cz$ (by proposition 7.1 page 152).
			let d denote the $\gcd(ab,c)$, therefore $d|c$ and $d| ab$. So, $\exists m,n$ \st $c=dm$ and $ab =dn$. 
			Consider, 
			\begin{align*}
				1 &= aw+cx\\
				b &= abw+cby 	& \text{(multiplying both sides by $b$.)}\\
				  &= dnw+dmby 	& \text{(substituting $c$ for $dm$ and $ab$ for $dn$.)}\\
				  &= d(nw+mby) 	& \text{(let $k$ denote $nw+mby$.)}
			\end{align*}
			Therefore $b = dk$. Now consider,
			\begin{align*}
				1 &= by + cz \\
				  &= dky+ dmz & \text{(substituting $c$ for $dm$, and $b$ for $dk$)}\\
				  &= d(ky+mz) \\				  
				  &= dj 	  & \text{(letting $j$ denote $ky+mz$)}
			\end{align*}
			This implies that $d|1$, therefore, d=1 (because 1 only has divisors -1, and 1; and 1 >-1). Therefore the $\gcd(ab,c)=1$, so the proposition holds. 
		\end{proof}
	 
	\end{enumerate}
		
	
	
	
\end{document}