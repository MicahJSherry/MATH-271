\documentclass{article}


\usepackage[english]{babel}
\usepackage{listings}
\usepackage[letterpaper,top=2.5cm,bottom=2.5cm,left=2.5cm,right=2.5cm,marginparwidth=1.75cm]{geometry}

\usepackage{amsmath,amsthm,amssymb}
\usepackage{bbold}
\usepackage{graphicx}
\usepackage{listings}
\usepackage{multicol}
\usepackage{setspace}
\usepackage{enumitem}


%common set symbols 
\newcommand{\Z}{\mathbb{Z}}
\newcommand{\N}{\mathbb{N}}
\newcommand{\R}{\mathbb{R}}
\newcommand{\Q}{\mathbb{Q}}
\newcommand{\C}{\mathbb{C}}
\newcommand{\nullset}{\varnothing}
\newcommand{\powerset}[1]{\mathcal{P}({#1})}


\newcommand{\st}{\text{ such that }}

\newcommand{\nitem}[1] % sets enumerate to a specific number {arg1}
{
	\setcounter{enumi}{#1}
	\addtocounter{enumi}{-1}
	\item
}

\title{MATH 271: Chapter 8 homework}
\author{Micah Sherry}
\singlespacing

\begin{document}
	\maketitle
	
	\begin{enumerate}
		\nitem{2} Prove that $\{6n | n \in \Z \} = \{2n | n \in \Z \} \cap \{3n | n \in \Z \}$
		\begin{proof}
			$(\subseteq)$ first we will show that $\{6n | n \in \Z \} \subseteq \{2n | n \in \Z \} \cap \{3n | n \in \Z \}$. \\
			Assume that $x \in \{6n | n \in \Z \}$. Therefore $x = 6n$ for some $n \in \Z$. So, $x = 2(3n)$ and because $3n \in \Z$, $x \in \{2n | n \in \Z \}$. Similarly $x=3(2n)$ and since $2n \in \Z$,  $x \in \{3n | n \in \Z \}$.
			Therefore $\{6n | n \in \Z \} \subseteq \{2n | n \in \Z \} \cap \{3n | n \in \Z \}$.
			
			$(\supseteq)$ Next we will show that $\{2n | n \in \Z \} \cap \{3n | n \in \Z \} \subseteq \{6n | n \in \Z \}$. \\
			Assume that $x \in \{2n | n \in \Z \} \cap \{3n | n \in \Z \}$. This implies that $2|x$ and $3|x$.
			this implies that $x = 6n$ for some $n \in \Z$ (because $2$ and $3$ share no factors other than $1$). Thus, $\{2n | n \in \Z \} \cap \{3n | n \in \Z \} \subseteq \{6n | n \in \Z \}$. Therefore the original equality holds. 
		\end{proof}
		
		\nitem{6} Suppose $A$,$B$ and $C$ are sets. Prove that if $A \subseteq B$, then $A -C \subseteq B-C$
		\begin{proof}
			Assume $A \subseteq B$ and let $x \in  A - C$. Therefore $x \in A$ and $x \not\in \C$. Since $A \subseteq B$ if $x \in A$ then $x \in B$. therefore $x \in B $ and $x \not\in C$. So, $x \in B-C$ and $A -C \subseteq B-C$. Therefore the original statement holds.
			
		\end{proof}
		
		\nitem{8} If $A$, $B$ and $C$ are sets, then $A \cup (B \cap C)$ = $(A \cap B)\cup  (A \cap C)$. 
		\begin{proof}
			
			Assuming  A, B, and C are sets.
			\begin{align*}
				A \cup (B \cap C) &= \{x| x \in A \vee x \in (B \cap C)\} & \text{(definition of union)}\\
				&= \{x| x \in A \vee x \in (x \in B \wedge x \in C)\} & \text{(definition of intersection)}\\
				&= \{x| x \in (A \vee x \in x \in B) \wedge(x \in A \vee x \in x \in C)\} & \text{(distributive property)}\\
				&= \{x| x \in (A \cup B) \wedge x \in( A \cup C)\} & \text{(definition of union)}\\
				&= \{x| x \in (A \cup B) \cap ( A \cup C)\} & \text{(definition of union)}\\
				&=(A \cap B)\cup  (A \cap C)
			\end{align*}
			Therefore the original equality holds.
		\end{proof}
		
		\nitem{10} If A and B are sets in a universal set U, then $\overline{A \cap B}= \overline{A} \cup \overline{B}$.
		\begin{proof} Assuming $A$ and $B$ are subsets of $U$. Then
			\begin{align*}
				\overline{A \cap B} &= \{x \in U| x \not\in A \cap B\} 
				& \text{(by definition of complement) } \\
				&= \{x \in U| \sim(x \in A \cap B)\} \\
				&= \{x \in U| \sim(x \in A \wedge x \in B)\} 
				& \text{(by definition of intersection)}\\
				&= \{x \in U| \sim(x \in A) \vee \sim(x \in B)\} 
				& \text{(by De Morgan's law $\sim(p \wedge q)= \sim p \vee \sim q$)} \\
				&= \{x \in U| x \not\in A \vee x \not\in B\}		\\
				&= \{x \in U| x \in \overline{A} \vee x \in \overline{B}\} 
				& \text{(by definition of complement) } \\
				&= \{x \in U| x \in \overline{A} \cup \overline{B}\} 
				& \text{(by definition of union)}\\
				&=\overline{A} \cup \overline{B}.
			\end{align*}
			Therefore the implication holds.  
		\end{proof}
		
		\nitem{16} If $A$,$B$ and $C$ are sets, then $A \times (B \cup C)= (A \times B) \cup (A \times C)$.
		\begin{proof}
			Assuming A, B, and C are sets. Then,
			\begin{align*}
				A \times (B \cup C) &= \{(x,y)| x \in A \wedge y \in (B \cup C)\} 		  & \text{(definition of Cartesian product)}\\
				&= \{(x,y)| x \in A \wedge (y \in B  \vee y \in  C)\} & \text{(definition of union)} \\
				&= \{(x,y)| (x \in A \wedge y \in B ) \vee(x \in A \wedge y \in  C)\}
				& \text{(the distributive property )}\\
				&= \{(x,y)| (x,y) \in A \times B  \vee (x,y) \in A \times C\} & \text{(definition of Cartesian product)}\\
				&= \{(x,y)| (x,y) \in (A \times B)  \cup (A \times C)\} & \text{(definition of union product)}\\
				&= (A \times B)  \cup (A \times C).
			\end{align*}
			Therefore the the implication holds.
		\end{proof}
		
		\nitem{20} Prove that $\{ 9^n| n \in \Q \} = \{ 3^n| n \in \Q \} $.
		\begin{proof}
			$(\subseteq)$ First we will show that $\{ 9^n| n \in \Q \} \subseteq \{ 3^n| n \in \Q \} $. \\
			Assume that $x \in \{ 9^n| n \in \Q \}$, then $x = 9^{n} =3^{2n}$ for some $n \in \Q$. Since $x = 3^{2n}$ and $2n \in \Q$, $x \in 9\{ 3^n| n \in \Q \} $. Therefore $\{ 9^n| n \in \Q \} \subseteq \{ 3^n| n \in \Q \} $. 
			
			$(\supseteq)$ Next, we will show that $\{ 3^n| n \in \Q \} \subseteq  \{ 9^n| n \in \Q \} $. \\
			Assume that $x \in \{ 3^n| n \in \Q \}$, then $x=3^{n} = 9^{\frac{n}{2}}$ for some $n \in \Q$. Since $x=9^{\frac{n}{2}}$ and $\frac{n}{2} \in \Q $, $x \in \{ 9^n| n \in \Q \}$. Therefore $\{ 3^n| n \in \Q \} \subseteq  \{ 9^n| n \in \Q \} $. Therefore since both subset relations hold so does the original equality.
		\end{proof}
		
		\nitem{26} Prove that $\{4k+5|k \in \Z\} = \{4k+1|k \in \Z\}$
		\begin{proof}
			$(\subseteq)$ First, we will show $\{4k+5|k \in \Z\} \subseteq \{4k+1|k \in \Z\}$.\\
			Assume that  $x \in \{4k+5|k \in \Z\} $ then $x =4k+5=4(k+1)+1$ for some $k \in \Z$.
			Since $x=4(k+1)+1$ and $k+1 \in \Z$. So, $x \in \{4k+1|k \in \Z\}$. 
			Therefore, $\{4k+5|k \in \Z\} \subseteq \{4k+1|k \in \Z\}$.
			
			$(\supseteq)$ Next, we will show $\{4k+1|k \in \Z\} \subseteq \{4k+5|k \in \Z\}$.\\
			Assume that  $x \in \{4k+1|k \in \Z\} $ then $x =4k+1=4(k-1)+5$ for some $k \in \Z$.
			Since $x=4(k-1)+5$ and $k-1 \in \Z$. So, $x \in \{4k+1|k \in \Z\}$.
			Therefore, $\{4k+5|k \in \Z\} \subseteq \{4k+1|k \in \Z\}$. Therefore since both subset relations hold so does the original equality.
		\end{proof}
	\end{enumerate}
	
\end{document} 