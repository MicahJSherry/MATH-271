\documentclass{article}


\usepackage[english]{babel}
\usepackage{listings}
\usepackage[letterpaper,top=2.5cm,bottom=2.5cm,left=2.5cm,right=2.5cm,marginparwidth=1.75cm]{geometry}

\usepackage{amsmath,amsthm,amssymb}
\usepackage{bbold}
\usepackage{graphicx}
\usepackage{listings}
\usepackage{multicol}
\usepackage{setspace}
\usepackage{enumitem}


%common set symbols 
\newcommand{\Z}{\mathbb{Z}}
\newcommand{\N}{\mathbb{N}}
\newcommand{\R}{\mathbb{R}}
\newcommand{\Q}{\mathbb{Q}}
\newcommand{\C}{\mathbb{C}}
\newcommand{\nullset}{\varnothing}
\newcommand{\powerset}[1]{\mathcal{P}({#1})}


\newcommand{\st}{\text{ such that }}

\newcommand{\nitem}[1] % sets enumerate to a specific number {arg1}
{
	\setcounter{enumi}{#1}
	\addtocounter{enumi}{-1}
	\item
}

\title{MATH 271: Chapter 9 homework}
\author{Micah Sherry}
\singlespacing

\begin{document}
	\maketitle
	
	\begin{enumerate}
		\nitem{2} For every natural number $n$, the integer $2n^2-4n+31$ is prime.
		\begin{proof}
			consider the counter-example $n = 31 $, then 
			\begin{align*}
				2n^2-4n+31 &= 2(31)^2-4(31)+31\\
				&= 31(2(31)-4+1)\\
				&= 31(59).
			\end{align*}
			Therefore $2n^2-4n+31$ is not prime for all $n$ (because it is divisible by 31 and 59). Thus the proposition is false.
		\end{proof}
		
		\nitem{6} If $A$, $B$, $C$, and $D$ are sets, then $(A \times B)\cap(C \times D)=(A \cap C) \times (B \cap D)$.
		\begin{proof}
			Assuming $A$, $B$, $C$, and $D$ are sets. Then,
			\begin{align*}
				(A \times B)\cap(C \times D) &= \{(x,y)| (x,y) \in (A \times B)\cap(C \times D)\}\\
				&= \{(x,y)| (x,y) \in (A \times B) \wedge (x,y) \in (C \times D)\} 
				& \text{(by definition of intersection)}\\
				&= \{(x,y)| (x \in A \wedge y \in B) \wedge (x \in C \wedge y \in D)\} 
				& \text{(by definition of Cartesian product)}\\
				&= \{(x,y)| x \in A \wedge y \in B \wedge x \in C \wedge y \in D\} 
				& \text{(by Associativity)}\\
				&= \{(x,y)| x \in A \wedge x \in C \wedge y \in B \wedge y \in D\} 
				& \text{(by Commutativity)}\\
				&= \{(x,y)| x \in A \cap C \wedge y \in B \cap D\}
				& \text{(by definition of intersection)}\\
				&= \{(x,y)| (x,y) \in (A \cap C) \times (B \cap D) \}
				& \text{(by definition of Cartesian product)}\\
				&=	(A \cap C) \times (B \cap D)
			\end{align*}
			Therefore the Equality holds, So the proposition is true.
		\end{proof}
		
		\nitem{16} If $A$ and $B$ are finite sets, then $|A \cup B| = |A|+|B|$.
		\begin{proof}
			Assuming $A$ and $B$ are finite sets. Consider the counter-example: $A=B$. 
			So, 
			\begin{align*}
				A \cup B &= \{x| x \in A \vee x \in B \} & \text{(by definition of union)}\\
				&= \{x| x \in A \vee x \in A \} & \text{(by substitution)}\\
				&= \{x| x \in A\} 					& \text{(since } p \vee p = p) \\
				&= A.
			\end{align*}
			Therefore $|A \cup B| = |A|$ and $|A|+|B| = 2|A|$. Thus the original equality does not hold, so the proposition is false.
		\end{proof}
		
		\nitem{18} If $a, b, c \in N$ then At least one of $a-b$, $a+c$ and $b-c$ is even.
		\begin{proof} 
			Assume for the sake of contradiction that $a-b$, $a+c$ and $b-c$ are all odd.
			then, $\exists k, j $ such that $a-b= 2k+1$ and $b-c= 2j+1$.
			So, $a = 2k +1+b$ and $c= -2j-1+ b$. Now consider 
			\begin{align*}
				a + c &= (2k +1+b) + (-2j-1+ b)\\
				&= 2k -2j +2b\\
				&= 2(k-j+b).
			\end{align*}
			Which is even and contradicts the Assumption that $a+c$ is odd. Thus the proposition holds.
		\end{proof}
		
		\nitem{20} There exist prime numbers $p$ and $q$ for which $p - q = 1000$.
		\begin{proof}
			Consider $p = 1013$ and $q = 13$. So, $p-q = 1013-13 = 1000$.  Since 1013 and 13 are prime the proposition is True.
		\end{proof}
		
		\nitem{28} Suppose $a,b \in \Z$ If $a|b$ and $b|a$, then $a = b$
		\begin{proof} Consider the counter-example: $a = -b$. This means $a = (-1)b$ which implies $b|a$. Similarly $b = (-1)a$ which implies $a|b$. Since $a|b$ and $b|a$ and $a \neq b$ therefore the proposition is false.
			
		\end{proof}
		
		\nitem{34} If $X \subseteq A \cup B$, then $X \subseteq A$ or $X \subseteq B$.
		\begin{proof}
			Consider the counter-example $A =\{1\}$, $B =\{2\}$, and $X=\{1, 2\}$. 
			So, $X \not\subseteq A$ and $X \not\subseteq B$.
			And notice $A \cup B = \{1, 2\}$ which implies $X \subseteq A \cup B$.
			Since $X \subseteq A \cup B$ and $X \not\subseteq A$ and $X \not\subseteq B$ the proposition is false. 
		\end{proof}
		
	\end{enumerate}
	
\end{document} 