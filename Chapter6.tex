\documentclass{article}


\usepackage[english]{babel}
\usepackage{listings}
\usepackage[letterpaper,top=2cm,bottom=2cm,left=2.5cm,right=2.5cm,marginparwidth=1.75cm]{geometry}

\usepackage{amsmath,amsthm,amssymb}
\usepackage{bbold}
\usepackage{graphicx}
\usepackage{listings}

\title{MATH 271: chapter 6 homework}
\author{Micah Sherry}

\newcommand{\ints}{\mathbb{Z}}
\newcommand{\reals}{\mathbb{R}}
\newcommand{\st}{\text{ such that }}
\newcommand{\nullset}{\varnothing}

\begin{document}
	\maketitle
	
	\section*{2. Suppose $n \in \ints$, If $n^2$ is odd then $n$ is odd} 
	
	Suppose for the sake of contradiction that $n^2$ is odd and $n$ is even.
	Since n is even $\exists k \in \ints \st n=2k$.
	Therefore,
	 \begin{align*}
	 	n^2 &= (2k)^2\\
	 	    &= (4k^2)\\
	 	    &= 2(2k^2).
	 \end{align*}
	Which implies n is even but, we assumed n is not even.
	Thus we have a contradiction, therefore the original statement holds.
	
	\section*{4. Prove that $\sqrt{6}$ is irrational}
	
	Suppose for the sake of contradiction that $\sqrt{6}$ is rational. 
	Then $\exists a,b \in \ints \st $ $a$ and $b$ have no common factors and $\sqrt{6}= \frac{a}{b}$ .
	So, 
	\begin{align*}
		6    &= \frac{a^2}{b^2}\\
		6b^2 &= a^2. 
	\end{align*}
	Which implies 6 divides $a^2$.
	

	Now we will show that if $a \in \ints$ and $6 \mid a^2$, then $6 \mid a$.
	Consider the contra-positive of this statement: If $6 \nmid a$, Then $6 \nmid a^2$
	Since 6 does not divide then $ \exists k \in \ints \text{and } r \in \{1,2,3,4,5\} $ \st $ a= 6k+r$ .
	So,
	\begin{align*}
		a^2 &= (6k+r)^2\\
		    &=36k^2+ 12kr + r^2\\
		    &= 6(6k^2 + 2kr) + r^2
	\end{align*}
	let $n = 6k^2 + 2kr$, So $a^2 = 6n + r^2$ 
	Now we need to check all Possible cases for $r$.
	
	\begin{table}[h]
		\centering
		\begin{tabular}{|c|c|c|}
			\hline
			$r$ & $r^2$ & $a^2$\\
			\hline
			1 & 1   & $6n+1$\\
			2 & 4   & $6n+4$\\
			3 & 9   & $6(n+1)+3$\\
			4 & 16  & $6(n+2)+4$\\
			5 & 25  & $6(n+4)+1$\\					
			\hline
		\end{tabular}
		\label{tab: cases for r}
	\end{table}
	Since in all cases 6 does not divide $a^2$ . Therefore the Contra-positive holds and thus 6 divides a.
	
	using the result of the proof above $a = 6k$ for some $k \in \ints$
	\begin{align*}
		6b^2 &= a^2. \\
			 &= 6(6k^2)^2.\\
		b^2 &= 6k^2. 
	\end{align*}
	Therefore $6$ divides $b^2$ and therefore divide b (by the above result). So, $a$ and $b$  share 6 as a factor which is a contradiction. Thus the original statement holds
	
	\section*{6. If $a,b \in \ints$ then $a^2-4b - 2 \neq 0$}
	Assume for the sake of contradiction that a and b are integers and $a^2-4b - 2 = 0$. Consider the case when $a=0$ and $b=0$ this would imply $0^2 -4(0)-2 =0 $ . Which is a contradiction. Thus the original statement holds.
	
	\section*{8. Suppose $a,b,c \in \ints$. If $a^2+b^2=c^2$ then $a$ or $b$ is even}
	Assume for the sake of contradiction that $a^2+b^2=c^2$, and $a$ and $b$ are odd. Then $\exists k,n\in \ints$ \st $a=2k+1$ and $b = 2n+1$. 	
	\begin{align*}
		c^2 &= a^2 + b^2\\
		 	&= (2k+1)^2 + (2k+1)^2\\
		 	&= 4k^2 + 4k + 1 + 4n^2 + 4n + 1 \\
		 	&= 4(k^2+n^2 + k + n) + 2
	\end{align*}
	let $m= k^2+n^2 + k + n$ Therefore $c^2$ = 4m+2.
	
	\textbf{Case 1:} c is even.
	Then $c= 2j$ for some $j \in \ints$. Then $c^2 = 4j^2$. Which is a contradiction because we have established that 4m + 2
	
	\textbf{Case 2:} c is odd.
	Then $c= 2j+1$ for some $j \in \ints$. Then $c^2 = 4j^2+4j+ 1 =4(j^2+j)+1$. Which is a contradiction because we have established that 4m + 2.
	
	Since both cases for c Lead to a contradiction The original statement holds.
	
	\section*{10. There exist no integers $a$ and $b$ such that  $21a +30b = 1$}
	Suppose for the sake of contradiction that $\exists a,b \st 21a+30b=1$. 
	\begin{align*}
		21a+30b   &=1\\
		3(7a+10b) &=1.
	\end{align*}
		This implies that 3 divides 1. Which is a contradiction. Thus the original statement holds 
	
	\section*{14. If A and B are sets then $A \cap (B-A)= \nullset$}
	Assume for the sake of contradiction that A and B are set $A \cap (B-A) \neq \nullset$. Then $\exists x \in A \cap (B-A)$
	\begin{align*}
		A \cap (B-A) &= \{y \mid y \in A  \text{ and } y \in(B-A)\} &\text{ (by definition of Union)} \\
		B-A          &= \{y \mid y \in B \text{ and } y \not\in A\} &\text{ (by definition of Set Subtraction)} \\
		A \cap (B-A) &= \{y \mid y \in A  \text{ and } y \in B \text{ and } y \not\in A\} 
	\end{align*}
	Therefore x must be an element of A and not an element of A. which Is a contradiction and thus the original statement is true.
	
\end{document}